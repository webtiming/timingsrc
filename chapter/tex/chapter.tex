%%%%%%%%%%%%%%%%%%%% author.tex %%%%%%%%%%%%%%%%%%%%%%%%%%%%%%%%%%%
%
% sample root file for your "contribution" to a contributed volume
%
% Use this file as a template for your own input.
%
%%%%%%%%%%%%%%%% Springer %%%%%%%%%%%%%%%%%%%%%%%%%%%%%%%%%%


% RECOMMENDED %%%%%%%%%%%%%%%%%%%%%%%%%%%%%%%%%%%%%%%%%%%%%%%%%%%
\documentclass[graybox]{svmult}

% choose options for [] as required from the list
% in the Reference Guide

\usepackage{mathptmx}       % selects Times Roman as basic font
\usepackage{helvet}         % selects Helvetica as sans-serif font
\usepackage{courier}        % selects Courier as typewriter font
\usepackage{type1cm}        % activate if the above 3 fonts are
                            % not available on your system
%
\usepackage{makeidx}         % allows index generation
\usepackage{graphicx}        % standard LaTeX graphics tool
                             % when including figure files
\usepackage{multicol}        % used for the two-column index
\usepackage[bottom]{footmisc}% places footnotes at page bottom

% see the list of further useful packages
% in the Reference Guide

\usepackage[justification=centering]{caption}

\makeindex             % used for the subject index
                       % please use the style svind.ist with
                       % your makeindex program

%%%%%%%%%%%%%%%%%%%%%%%%%%%%%%%%%%%%%%%%%%%%%%%%%%%%%%%%%%%%%%%%%%%%%%%%%%%%%%%%%%%%%%%%%

\begin{document}

\title*{Media Synchronization on the Web}
% Use \titlerunning{Short Title} for an abbreviated version of
% your contribution title if the original one is too long
\author{Ingar M. Arntzen, Nj{\aa}l T. Borch and Fran\c{c}ois Daoust}
% Use \authorrunning{Short Title} for an abbreviated version of
% your contribution title if the original one is too long
\institute{Ingar M. Arntzen \at Norut Northern Research Institute, Troms{\o}, Norway \email{ingar.arntzen@norut.no}
\and Nj{\aa}l T. Borch \at Norut Northern Research Institute, Troms{\o}, Norway \email{njaal.borch@norut.no}
\and Fran\c{c}ois Daoust \at World Wide Web Consortium (W3C), Paris, France \email{fd@w3.org}}

\maketitle

% abstract
\abstract{
Proliferation of user devices makes synchronized, multi-device, multimedia
a reasonable expectation, and the Web platform should be a natural home
for such media experiences. Unfortunately though, it lacks a vital ingredient; support for multi-device 
media synchronization. This chapter presents the motion model, a
general approach to media synchronization on the Web. The motion model targets
simplicity and Web availability, emphasizing 1) that media synchronization
must be easy for Web developers, and 2) that media synchronization on the Web
must work wherever and whenever the Web works. In the motion model, media
clock and media controls are merged into a single concept, motion, and made
available as an online resource. Media synchronization is achieved by
connecting distributed media components to the same online motion. The
feasibility of this approach is confirmed by evaluating synchronization of
HTML5 media elements connected to online motion. Echoless synchronization of
audio and video is routinely produced across different Web browsers,
architectures and devices, making no assumptions except connectivity.
Importantly though, the value promise of the motion model is not limited to
synchronization alone. In particular, the extreme flexibility of the motion
model sets it up as a foundation for Web-based, multi-device, multimedia.
Though the motion model is designed explicitly for the Web, its utility is not
limited to the Web. Any connected device may connect to an online motion,
opening for synchronized media experiences across Web and native platforms.
The motion model has been suggested for Web standardization by the W3C multi-device 
timing community group.

}

% introduction
\section{Introduction}
\label{sec:intro}
%The proliferation of user devices as well as the eagerness of people to use
them, make seamless multi-device media experiences a natural vision in the
media industry. Such media experiences might for instance exploit big screens
for video content, whereas laptops, tablets and smartphones may be useful for
presenting associated content, such as infographics, backstory, foreign
language resources, control features or interactivity. People might also
expect much flexibility regarding the configuration of such experiences. Some
might meticulously define their own setups for a favourite event, whereas
others will prefer that media experiences automatically adapt to exploit
available devices. It must also be possible to shift media experiences
dynamically between devices, and adapt to new circumstances. For instance,
while driving, users on a strict network quota might want a live TV show to
adapt by dropping the video stream all toghether, while keeping the audio and
the secondary content. In addition, users always need the ability to control
their media sessions; e.g. pausing a live experience and resuming it at a more
convenient time, perhaps the following day on a different device. Importantly,
controls such as pause, rewind, fastforward or \emph{go to next scene} should
generally apply to all devices. In short, in this new multi-device world,
people wish to interact with media experiences through any device or multiple
devices, and media providers must create flexible media products capable of
exploiting these circumstances well. In particular, media products must support
radical transformations when circumstances change in significant ways. 


The Web platform appears to be a great match for this new multi-device world. It
is a powerful platform for multimedia, with global reach, powerful backend
services. Web browsers have a rich set of sensors for capture, interactive
capabilities, and a variety of frameworks for visualization and media
presentation. The modularity, interoperability and programmability of the Web
platform is also an excellent foundation for highly flexible media products.
Finally, the Web is already multi-device and global in nature. People may
interact with their online mailbox through any device, or using multiple
devices at the same time. Furthermore, they expect their online mailbox to be
the same, independent of which device is used to access it, or where they
might be at the time. Clearly, this basic promise of the Web is equally
attractive in the context of multi-device media experiences.
 
Unfortunately though, the Web lacks a crucial ingredient. Multi-device media
experiences (unlike mailboxes) require precise, distributed media
synchronization, and the Web platform provides no support for this. To fill
this gap, this chapter presents the motion model, a general approach to
media synchronization on the Web. The motion model is designed with the
following key objectives.

\begin{itemize}
\item{\emph{Global:} Media synchronization on the Web must be multi-device 
and global in nature, simply because the Web is multi-device and global in nature.}
\item{\emph{Web-availability:} If a Web browser is able to load a Web page, 
it should also be able to synchronize correctly.}
\item{\emph{Simplicity:} Media synchronization should be easy for Web developers. 
In particular, developers should be shielded from complexities of distributed synchronization.}
\end{itemize}

The key idea of the motion model is very simple: Media playback must be
represented as an online resource. This allows Web pages globally to
synchronize presentations relative to a shared playback state. Media
synchronization between Web pages is simply a consequence of connecting to the
same playback resource.

Though the primary purpose of the motion model is synchronization, the model
has implications that go far beyond this topic. In particular, the motion
model inverts the relationship between synchronization and media.
Traditionally, synchronization is external to media and may be applied to
media, if needed. In contrast, the motion model defines synchronization as an
integral part of media. As such, the motion model implies a new media model
particularly suited for highly flexible multi-device multimedia experiences.

The chapter is structured as follows: Sect.~\ref{sec:mediasync} defines media
synchronization as a term and briefly presents common challenges for synchronization on the Web. The motion
model is presented in Sect.~\ref{sec:model}, followed by
Sect.~\ref{sec:properties} which highlights key properties of the implied
media model. Sect.~\ref{sec:web} surveys relevant Web technologies, relating
them to the motion model.
Sect.~\ref{sec:motion},~\ref{sec:motionsync} and~\ref{sec:compsync} give a more
thorough introduction to the motion concept, to distributed synchronization of
motions, and to the challenge of synchronizing media components. Evaluation is
presented in Sect.~\ref{sec:eval}. Sect.~\ref{sec:standard} briefly references
the standardization initiative made by the Multi-device Timing Community
Group~\cite{mtcg}, before conclusions are given in Sect.~\ref{sec:concl}.


% abbreviations
\section{Abbreviations}
\label{sec:abbr}
%\begin{itemize}
\item{\textbf{URL:} universal resource locator}
\item{\textbf{HTML5:} hypertext markup language, version 5}
\item{\textbf{W3C:} World Wide Web Consortium}
\item{\textbf{HTTP:} hypertext transfer protocol}
\item{\textbf{AV:} audio and/or video}
\item{\textbf{DOM:} document object model}
\item{\textbf{UI:} user interface}
\item{\textbf{API:} application programmer interface}
\item{\textbf{IP:} Internet protocol}
\item{\textbf{UDP:} user datagram protocol}
\item{\textbf{TCP:} transmission control protocol}
\item{\textbf{NTP:} network time protocol}
\item{\textbf{PTP:} precision time protocol}
\item{\textbf{RTT:} network round-trip time}
\item{\textbf{WAN:} wide area network}
\item{\textbf{EDGE:} Enhanced Data rates for GSM Evolution, typically lowest level of Internet connectivity for smart phones.}
\item{\textbf{JSON:} JavaScript object notation}
\item{\textbf{GPS:} global positioning system}
\item{\textbf{UTC:} coordinated universial time}
\item{\textbf{XML:} extendible markup language}
\end{itemize}


% terminology
\section{Terminology}
\label{sec:terms}
%\begin{itemize}
\item{\textbf{epoch:} clock counting seconds (or milliseconds) since Jan 1. 1970}
\item{\textbf{timeline:} logical axis for media playback}
\item{\textbf{media clock:} clock representing media playback state, e.g. media offset}
\item{\textbf{media controls:} operations that alter a media clock, e.g. play/pause}
\item{\textbf{timed data:} data tied to points or intervals on the timeline}
\item{\textbf{continuous media:} ordered sequence of media objects, e.g. audio or video frames}
\item{\textbf{discrete media:} not continuous media, e.g media objects which overlap on the timeline, or are distributed scarcely or non-uniformly}
\item{\textbf{JavaScript:} programming language of the Web}
\item{\textbf{browser context:} JavaScript runtime hosted by a Web browser}
\item{\textbf{iframe:} Webpage within a Webpage, with its own browser context}
\item{\textbf{media component:} anything from a simple $<div/>$ element to a highly sophisticated media player or framework. Media components have \emph{media data} and \emph{media clock}. Media data
is organized according to a timeline, and media clock represents playback along
this timeline. Through its \emph{user interface (UI)}, media components
express \emph{media output} (e.g. pixels on screen, audio, vibration) and
receive \emph{media input} (e.g. key-presses, touch or mouse-events, camera, mic).}
\end{itemize}

% media synchronization
\section{Media Synchronization}
\label{sec:mediasync}
Dictionary definitions of \emph{media synchronization} typically refer to
presentation of multiple instances of media at the same time. A related term
is \emph{media orchestration}, possibly emphasizing more the importance of
media control and planned scheduling of media playback. Similarly, we have
used the term \emph{media timing} to highlight the universal utility of clocks
in digital media, for capture, control, synchronization, timeshifting,
scheduling and playback. In this chapter we use the term \emph{media
synchronization} in a broad sense, as a synonym to \emph{media orchestration}
and \emph{media timing}. We also limit the definition in two regards:

\begin{itemize}
\item{Media synchronization on the Web is clock-based. The latencies and heterogeneity of the Web environment requires a clock-based approach for acceptable synchronization quality.}
\item{Media synchronization involves one media instance and a clock. The term relative synchronization is reserved for comparisons between media instances.}
\end{itemize}

\subsection{Challenges}

\begin{table}
\centering
\caption{Common challenges for media synchronization on the Web.}
\label{tab:challenges}
\setlength{\tabcolsep}{10pt}
\begin{tabular}{cc}
\hline\noalign{\smallskip}
Challenge & Use-cases \\
\noalign{\smallskip}\svhline\noalign{\smallskip}
across media sources & multi-angle video, ad-insertion\\
across media types & video, WebAudio, animated map\\
across iframes & video, timed ad-banner\\
across tabs, browsers, devices & split content, interaction\\
across platforms & Web, native, broadcast\\
across people and groups & collaboration, social\\
across Internet & global media experiences\\
\noalign{\smallskip}\hline\noalign{\smallskip}
\end{tabular}
\end{table}

Media synchronization has a wide range of use-cases on the Web, as illustrated
by table~\ref{tab:challenges}. Well known use-cases for synchronization within a
single Web page include multi-angle video, accessibility features for video,
ad-insertion, as well as media experiences spanning different media types,
media frameworks, or iframe boundaries. Synchronization across Web pages allow
Web pages to present alternative views into a single experience, increasing
possibilities further. Popular use-cases in the home environment involve
collaborative viewing, multi-speaker audio, or big screen video synchronized
with related content on handheld devices. Use-cases towards the end of the
list target global scenarios, such as distributed capture and synchronized Web
presentations for a global audience.

\subsection{Approach}

The challenges posed by all these use-cases may be very different in terms of
complexity, requirements for precision, scale, infrastructure and more. Yet,
we argue that a single solution is needed. Implementing specific solutions for
specific use-cases is very expensive and time-consuming, and lays heavy
restrictions on reusability. Even worse, circumstances regarding
synchronization may change dynamically during a media session. For instance, a
smartphone involved in synchronization over the local network, will have to
switch approach for synchronization once the user leaves the house, or
switches from WiFi to the mobile network. Crucially though, by solving media
synchronization across Internet, all challenges listed above are solved by
implication. For instance, if video synchronization is possible across Web
pages on the Internet, then synchronizing two videos within the same Web page
is just a special case. It follows that the general solution to media
synchronization on the Web is distributed and global in nature. Locality may
be exploited for synchronization, yet only as optimization.


% motion model
\section{The Motion Model}
\label{sec:model}
The primary objectives of the motion model are global synchronization,
\emph{Web availability} and simplicity for Web developers. Global
synchronization implies media synchronization across the Internet. Web
availability means that no additional assumptions can be introduced for media
synchronization. If a Web browser is able to load an online Web page, it
should also be able to synchronize correctly. The model proposed for this can
be outlined in three simple steps:

\begin{itemize}
\item{Media clock and media controls are encapsulated in one concept, and
represented as a stateful resource. This chapter uses the term \emph{motion}\footnote{\emph{motion} as in \emph{motion pictures}. \emph{Moving through media} still remains a good way to conceptualize media experiences, not least as media experiences become virtual and immersive.
} for this
concept.} 
\item{A \emph{motion} is an online resource, implying that it is hosted by a
server and identifiable by a \emph{Universal Resource Locator (URL)}.}
\item{\emph{Media components}\footnote{\emph{media component:} anything from a simple DOM element with text, to a highly sophisticated media player or multimedia framework.
} synchronize themselves relative to online motions.}
\end{itemize}

According to the model, media synchronization should be a consequence of
connecting multiple media components to the same online motion. This way, rich
synchronized multi-device presentation may be crafted by connecting relevant
media components to the same online motion, as illustrated in Fig.~\ref{fig:model}.

\begin{figure}[h]
%\sidecaption
\centering
\includegraphics[scale=.4]{fig/motion-model.png}
\caption{Media components on three different devices (A,B,C), all connected to an online motion
(red circle). Media control requests (e.g. pause/resume) target the online motion and are transmitted across the Internet (light blue cloud). The corresponding state change is
communicated back to all connected media components. Each media component
adjusts its behaviour independently.}
\label{fig:model}
\end{figure}

Importantly, the practicality of the motion model depends on Web developers
being shielded from the complexities of distributed synchronization. This is
achieved by having a \emph{timing object} locally in the Web browser. The
timing object acts as an intermediary between media components and online
motions, as illustrated by Fig.~\ref{fig:model-2}. This way, the challenge of
media synchronization is divided in two parts.

\begin{itemize}
\item{\emph{motion synchronization}: timing object precisely synchronized with online motion (Internet problem).}
\item{\emph{component synchronization}: media component precisely synchronized with timing objects (local problem).} 
\end{itemize}


\begin{figure}[h]
%\sidecaption
\centering
\includegraphics[scale=.4]{fig/motion-model-2.png}
\caption{Timing objects (red unfilled circles) mediate access to online motion. Timing objects may be shared by independent media components within the same browing context.}
\label{fig:model-2}
\end{figure}

\emph{Motion synchronization} ensures that timing objects connected to the
same online motion are kept precisely synchronized. The logic required for
motion synchronization could be supported by Web browsers natively (if
standardized), or imported into Web pages as a third party JavaScript
library. Motion synchronization is outlined in Section~\ref{sec:motionsync}.

\emph{Component synchronization} implies that a media component continuously
strives to synchronize its activity relative to a timing object. As such,
component synchronization is a local problem. Media components always
interface with timing objects through a well defined \emph{Application Programmer Interface (API)} (see
Section~\ref{sec:motionapi}). Examples of component synchronization are provided in Section~\ref{sec:avsync} and Section
~\ref{sec:sequencer}.



% key properties
\section{Key properties}
\label{sec:properties}
%The value promise of the motion model is not limited to synchronization. This
section presents key properties of the media model that is implied by the
motion model.

\subsubsection{Decoupling media components from online motions} 

Motion objects in the Web browser decouple media components from online
motions. For Web developers, this means that the problem of media
synchronization is reduced to component synchronization, essentially an
exercise in regular Web programming (see Sect.~\ref{sec:compsync}). This way,
globally synchronized media components and Web experiences may be built by
practitioners with limited competence in distributed synchronization.
Furthermore, a standardized API for motion would allow media products to
easily exploit the combined power of very different media components and media
technologies. It would also ensure reusability of media components, for
example across applications, across live and on-demand media, or for offline
usage. Offline synchronization may be supported by a local motion object (i.e.
not connected to an online motion). A standardized API for motion would also
open up for alternative implementations of motion synchronization.

\subsubsection{Decoupling motion from media content}

The motion model describe how media components connect to online motion, yet
it does not include any references to media content. This is on purpose. In
the motion model, online motion is separate from content transfer, which is
considered an independent problem. In particular, the service hosting online
motion is not required to host any media content at all\footnote{Online motion
should preferably be hosted by a dedicated service, thus avoiding disturbances
resulting from unrelated server-side processing or bandwidth consumption.}.
Instead, media components will fetch their media content from other resources
or services. Strict isolation of online motion ensures that no restrictions
are put on media components regarding choice of media formats or transport
mechanisms. By implication, the motion model supports synchronization across
any media format or transport protocol\footnote{Media content must refer to a
timeline, in order for media components to be able to synchronize it with
online motion.}.

\subsubsection{Decoupling media components from each other}

The motion model adopts the client-server architecture of the Web. This allows
media components to remain independent of other media components. This is
important. For example, a failing media component should not be allowed to
halt an entire distributed media experience. Moreover, independence of media
components yields other important properties, such as flexibility,
extensibility, dynamism, and reduced complexity. In addition, independence of
media components simplifies development and testing, as well as integration of
media products across different technologies and platforms.

\subsubsection{Online media control}

In the motion model, Web availability is achieved by representing media
control and an online resource. Online media control implies that media
control requests (e.g. pause/resume) target the online motion, and only
affects media components after the result is communicated
back\footnote{Favouring responsiveness over synchronization, it would be
possible for a media component to implement the effects of a media control
request speculatively, before sending it to the online motion. This feature is
not a part of the model, but may be implemented in application code if needed.
}. Importantly, this means that synchronized media experiences based on online
media control are remote controllable by design, from any connected device.
This allows multi-device media experiences to make control features available
from multiple devices or views, as opposed to depending on a single, dedicated
control device. By adding access control to online motions, media control
privileges may be appropriately restricted.


\subsubsection{Dynamic participation}

The Web environment is characterized by high \emph{churn}, i.e. Web clients
may be expected to disconnect and reconnect at any time. To support this,
clients must be able to join and leave a synchronized media presentation
without disrupting other clients, or the presentation itself. The client-
server architecture of the Web makes this easy. For instance, reloading a
synchronized Web page is simply a matter of disconnecting from, and
reconnecting to, an online motion. Note also that online motions do not depend
on connected clients for their existence. If an online motion is requested to
play, it will continue to do so until requested otherwise, even if all media
components have disconnected. This allows a Web client to occasionally peek
into an ongoing media presentation, even if no other clients are
watching\footnote{Whether the media presentation exists or not, when no media
components are connected to the online motion, remains a philosophical
question.}. This is what is typically expected from live media productions.


\subsubsection{Symmetri}

Asymmetric master-slave synchronization is a common pattern in media
synchronization. The pattern implies that a set of media components (slaves)
are synchronized relative to a specific media component (master). In the
motion model all media components are slaves, and the motion is the master.
This way, the motion model avoids added complexities of the master-slave
pattern, and provides a symmetric model where each media component may request
control via the motion. If asymmetry is more appropriate for a given
application, this may easily be emulated. For instance, applications may
ensure that only a specific media component may issue control requests to the
motion.

\subsubsection{Internal Synchronization}

At present, external synchronization is the common pattern on the Web, where
media components are coordinated by manipulating their control primitives.
Unfortunately, media components on the Web have not been particularly designed
with synchronization in mind. Also, external synchronization generally gets
more difficult as the number of components increases. Heterogeneity in media
types and control interfaces complicate matters further. In contrast, internal
synchronization is the pattern implied by the motion model, where media
components synchronize themselves relative to an external motion. In this
pattern, precise media synchronization may be implemented with unrestricted
access to the internal state and capabilities of the particular media
component. Furthermore, the synchronization task is shifted from Web
developers using the component, to the author of the media component. This
ensures that the problem may be solved once, instead of repeatedly by
different application developers.


\subsubsection{Single-source media components}

Dealing with multiple media assets, one approach is to collect them into
standardized file or container formats. This approach allows co-transport of
multiple media sources (as one source), and requires a media player supporting
the particular transport protocol and media format. The motion model enables
the opposite approach, where each media component is dedicated to a single
content source. This removes the need for bundling content sources, and allows
an appropriate transport mechanism to be selected for each content source. For
instance, subtitle-tracks do not have to be bundled with video files, but may
instead be represented as online resources. This way, media providers may even
edit them real-time, while they are being displayed for content viewers.
Importantly though, the motion model leaves the design choices regarding media
transport and media sources entirely to developers. Existing container-based
media frameworks are fully compatible with the motion model, provided only
that they may be controlled by motion (see Sect.~\ref{sec:integration}).

\subsubsection{Single-purpose media components}

In the motion model, media components may also be specialized with respect to
purpose. For instance, a single media component may implement interactive
controls for motion, thereby relieving other media components from the
complexity of this function. This encourages a pattern where media components
are designed for specific roles in an application, e.g. controllers, viewers
and editors, and combined to form the full functionality. Even though these
media components are independent, appropriate layout and styling may hide
this, giving the impression of a tightly integrated media product.



% state of the Web
\section{State of the Web}
\label{sec:web}
%With temporal interoperability established as a goal for the Web, this section
surveys current abilities and limitations of the Web with respect to
media synchronization. The Web platform\footnote{In this chapter, the Web is
seen through the eyes of an end user browsing the Web with his/her favorite
\emph{user agent} in 2017.} is composed of a series of technologies centered
around the \emph{Hypertext Markup Language (HTML)}. These technologies have
been developed over the years and have grown steadily since the advent of
\emph{HTML5}~\cite{html5}, allowing Web applications to access an  ever-increasing 
pool of features such as local storage, geo{\-}location, peer-to-peer
communications, notifications, background execution, media capture, and more.
This section focuses on Web technologies that produce or
consume timed data, and highlights issues that arise when these technologies
are used or combined with others for synchronization purposes. These issues
are classified and summarized at the end of the section. Please note that this
section is written early 2017. It references technologies that are still under
development.

\subsection {HTML}
\label{sec:web-html}

First versions of the HTML specification (including HTML3.2~\cite{html32}) were
targeting static documents and did not have any particular support for timed playback. 
HTML5 introduced the Audio and Video media elements to add support for
\mbox{audio} and video data playback. Web applications may control the playback of
media elements using commands such as \emph{play} or \emph{pause} as well as
properties such as \mbox{\emph{currentTime}} (the current media offset)
and \emph{playbackRate} (the playback speed). In theory, this should be enough
to harness media element playback to any synchronization logic that authors
may be willing to implement. However, there are practical issues:

\begin{enumerate}

\item{

The playback offset of the media element is measured against a media clock,
which the specification defines as: \emph{user-agent defined, and may be media
resource-dependent, but [which] should approximate the user's wall clock.} In
other words, HTML5 does not impose any particular clock for media playback.
One second on the wall clock may not correspond to one second of playback, and
the relationship between the two may not be linear. Two media elements playing
at once on the same page may also follow different clocks, and thus media
offset of these two media elements may diverge over time even if playback was
initiated at precisely the same time.

}

\item{

HTML5 gives no guarantee about the latency that the software and the hardware
may introduce when the play button is pressed, and no compensation is done to
resorb that time afterwards.

}

\item{

The media clock in HTML5 automatically pauses when the user agent needs to
fetch more data before it may resume playback. This behavior matches the
expectations of authors for most simple media use cases. However, more advanced
scenarios where media playback is just a part of a larger and potentially
cross-device orchestration would likely require that the media clock keeps
ticking no matter what.

}

\item{

The \emph{playbackRate} property was motivated by the fast forward and rewind
features of \emph{Digital Video Disc (DVD)} players and previously
\emph{Videocassette Recorders (VCR)}. It was not meant for precise control of
playback velocity on the media timeline.

}

\end{enumerate}

To address use cases that would require synchronized playback of media
elements within a single page, for instance to play a sign language track as
an overlay video on top of the video it describes, HTML5 introduced the
concept of a \emph{media controller}~\cite{mediacontroller}. Each media
element can be associated with a media controller and all the media elements
that share the same media controller use the same media clock, allowing
synchronized playback. In practice though, browser vendors did not implement
media controllers and the feature was dropped in HTML5.1~\cite{html51}. It is
also worth noting that this mechanism was restricted to media elements and
could not be used to orchestrate scenarios that involved other types of timed
data.

While sometimes incorrectly viewed as a property of the JavaScript language,
the \emph{setTimeout}, \emph{setInterval} and other related timer functions,
which allow apps to schedule timeouts, are actually methods of the
\emph{window} interface, defined in HTML5. These methods take a timeout
counter in milliseconds, but the specification only mandates that Web browsers
wait until at least this number of milliseconds have passed (and only provided
the Web page has had the focus during that time). In particular, Web browsers
may choose to wait a further arbitrary length of time. This allows browsers to
optimise power consumption on devices that are in low-power mode. Even if
browsers do not wait any further, the event loop may introduce further delays
(see Section~\ref{sec:eventloop}). Surprisingly, browsers also fire timers too
early on occasion. All in all, the precision of timeouts is not guaranteed on
the Web, although experience shows that timeouts are relatively reliable in
practice.

\subsection{SMIL and Animations}
\label{sec:smil}

Interestingly, one of the first specifications to have been published as a Web
standard after HTML3.2~\cite{html32}, and as early as 1998, was the
\emph{Synchronized Multimedia Integration Language (SMIL)} 1.0
specification~\cite{smil1}. SMIL allowed integrating a set of independent
multimedia objects into a synchronized multimedia presentation. SMIL 1.0 was
the first Web standard to embed a notion of timeline (although it was only
implicitly defined). The specification did not mandate precise synchronization
requirements: \emph{the accuracy of synchronization between the children in a
parallel group is implementation-dependent.} Support for precise timing has
improved in subsequent revisions of SMIL, now in version 3.0~\cite{smil3}.

No matter how close to HTML it may be, SMIL appears to Web application
developers as a format on its own. It cannot simply be added to an existing
Web application to synchronize some of its components. SMIL has also never
been properly supported by browsers, requiring plugins such as
RealPlayer~\cite{realplayer}. With the disappearance of plugins in Web
browsers, authors are left without any simple way to unleash the power of SMIL
in their Web applications.

That said, SMIL 1.0 sparked the SMIL Animation
specification~\cite{smilanimation} in 2001, which builds on the SMIL 1.0
timing model to describe an animation framework suitable for integration with
\emph{Extensible Markup Language (XML)} documents. SMIL Animation has notably been incorporated in the Scalable
Vector Graphics (SVG) 1.0 specification~\cite{svg}, published as a Web
standard immediately afterwards. It took many years for SVG to take over
Flash~\cite{flash} and become supported across browsers, with the notable
exception of SMIL animations, which \emph{Microsoft}~\cite{microsoft} never
implemented, and which \emph{Google}~\cite{google} now intends to drop in
favor of \emph{CSS Animations} and of the \emph{Web Animations specification}.

While still a draft when this book is written, Web Animations~\cite{webanimation} appears as
a good candidate specification to unite all Web animation frameworks into one,
with solid support from \emph{Mozilla}~\cite{mozilla}, Google and now Microsoft. It introduces the
notion of a \emph{global clock}:
\begin{quote} a source of monotonically increasing time values unaffected by
adjustments to the system clock. The time values produced by the global clock
represent wall-clock milliseconds from an unspecified historical moment.
\end{quote}
The specification also defines the notion of a \emph{document timeline}
that provides time values tied to the global clock for a particular document.
It is easy to relate the global clock of Web Animations with other clocks
available to a Web application (e.g. the \emph{High Resolution Time} clock mentioned
in Section~\ref{sec:hrt}). However, the specification acknowledges that the setup of some
animations \emph{may incur some setup overhead}, for instance when the user agent
delegates the animation to specialized graphics hardware. In other words, the
exact start time of an animation cannot be known a priori.



\subsection{DOM Events}
\label{sec:domevents}

The ability to use scripting to dynamically access and update the content,
structure and style of documents, was developed in parallel to HTML, with
\emph{ECMAScript} (commonly known as JavaScript), and the publication of the
\emph{Document Object Model (DOM) Level 1} standard in 1998~\cite{dom1}. This
first level did not define any event model for HTML documents, but was quickly
followed by \emph{DOM Level 2}~\cite{dom2} and in particular the DOM Level 2
Events standard~\cite{domevents} in 2000. This specification defines: \emph{a
platform- and language-neutral interface that gives to programs and scripts a
generic event system}.

DOM events feature a \emph{timeStamp} property used to specify the time relative to
the epoch at which the event was created. DOM Level 2 Events did not mandate
that property on all events. Nowadays, DOM Events, now defined in the DOM4
standard~\cite{dom4}, all have a timestamp value, evaluated against the system
clock.

The precision of the timestamp value is currently limited to milliseconds, but
Google has now switched to using higher resolution timestamps associated with
the \emph{high resolution clock} (see Section~\ref{sec:hrt}). On top of
improving the precision down to a few microseconds, this change also means
that the \emph{monotonicity} of timestamp values can now be guaranteed.
\emph{Monotonicity} means that clock values are never decreasing. This change
will hopefully be included in a future revision of the DOM standard and
implemented across browsers.



\subsection{The Event Loop}
\label{sec:eventloop}

On the Web, all activities (including \emph{events, user interactions,
scripts, rendering, networking}) are coordinated through the use of an
\emph{event loop}\footnote{There may be more than one event loop, more than
one queue of tasks per event loop, and event loops also have a micro-task
queue that helps prioritizing some of the tasks added by HTML algorithms, but
this does not change the gist of the comments contained in this section.},
composed of a queue of tasks that are run in sequence. For instance, when the
user clicks a button, the user agent queues a task on the event loop to
dispatch the \emph{click} event onto the document. The user agent cannot
interrupt a running task in particular, meaning that, on the Web, all scripts
run to completion before further tasks may be processed.

The event loop may explain why a task scheduled to run in 2 seconds from now
through a call to the \emph{setTimeout} function may actually run in 2.5 seconds
from now, depending on the number of tasks that need to run to completion
before this last task may run. In practice, HTML5 has been carefully designed
to optimize and prioritize the tasks added to the event loop, and the
scheduled task is unlikely to be delayed by much, unless the Web application
contains a script that needs to run for a long period of time, which would
effectively freeze the event loop.

Starting in 2009, the Web Workers specification~\cite{webworkers} was developed to allow
Web authors to run scripts in the background, in parallel with the scripts
attached to the main document page, and thus without blocking the user
interface and the main event loop. Coordination between the main page and its
workers uses message passing, which triggers a \emph{message} event on the event
loop.

Any synchronization scenario that involves timed data exposed by some script
or event logic will de facto be constrained by the event loop. In turn, this
probably restricts the maximum level of precision that may be achieved for
such scenarios. Roughly speaking, it does not seem possible to achieve less
than one millisecond precision on the Web today if the event loop is involved.


\subsection{High Resolution Time}
\label{sec:hrt}

In JavaScript, the \emph{Date} class exposes the system clock to Web
applications. An instance of this class represents a number of milliseconds
since January 1., 1970 UTC. In many cases, this clock is a good enough
reference. It has a couple of drawbacks though:

\begin{enumerate}

\item{

The system clock is not monotonic and it is subject to adjustments. There is
no guarantee that a further reading of the system clock will yield a greater
result than a previous one. Most synchronization scenarios need to rely on the
monotonicity of the clock. 

}
\item {

Sub-millisecond resolution may be needed in some
cases, e.g. to compute the frame rate of a script based animation, or to
precisely schedule audio cues at the right point in an animation.

}

\end{enumerate}

As focus on the Web platform shifted away from documents to applications and
as the need to improve and measure performance arose, a need for a better
clock for the Web that would not have these restrictions emerged. The High
Resolution Time specification~\cite{hrt1} defines a new clock,
\emph{Performance.now()}, that is both guaranteed to be monotonic and accurate
to 5 microseconds, unless the user agent cannot achieve that accuracy due to
software or hardware constraints. The specification defines the time origin of
the clock, which is basically the time when the \emph{browsing context} (i.e.
browser Window, tab or iFrame) is first created. The very recent High
Resolution Time Level 2 specification~\cite{hrt2} aims to expose a similar
clock to background workers, and provide a mechanism to relate times between
the browsing context and workers.

It seems useful to point out that the 5 microseconds accuracy was not chosen
because of hardware limitations. It was rather triggered by privacy concerns
as a way to mitigate so called cache attacks, whereby a malicious Web site
uses high resolution timing data to fingerprint a particular user. In
particular, this sets a hard limit to precision on the Web, that will likely
remain stable over time.


\subsection{Web Audio API}
\label{sec:webaudio}

At about the same time that people started to work on the High Resolution Time
specification, Mozilla and Google pushed for the development of an API for
processing and synthesizing audio in Web applications. The Web Audio API draft
specification~\cite{webaudio} is already available across browsers. It builds upon an
audio routing graph paradigm where audio nodes are connected to define the
audio rendering.

Sample frames exposed by the Web Audio API have a \emph{currentTime} property that
represents the position on the Audio timeline, according to the clock used to
generate the audio stream. In other words, the hardware clock of the underlying
sound card. As alluded to in the specification, this clock \emph{may not be
synchronized with other clocks in the system}. In particular, there is little
chance that this clock be synchronized with the High Resolution Time clock,
the global clock of Web Animations, or the media clock of a media element.

The group that develops the Web Audio API at W3C investigated technical
solutions to overcome these limitations. The API now exposes the relationship
between the audio clock and the high resolution clock, coupled with the
latency introduced by the software and hardware, so that Web applications may
compute the exact times at which a sound will be heard. This is particularly
valuable for cross-device audio scenarios, but also allows audio to
be output on multiple sound cards at once on a single device.




\subsection{Media Capture}
\label{sec:capture}

W3C started to work on the Media Capture and Streams
specification~\cite{capture} in 2011. This specification defines the notions
of \emph{MediaStreamTrack}, which \emph{represents media of a single type that
originates from one media source} (typically video produced by a local camera)
and of \emph{MediaStream}, which is a group of loosely synchronized
MediaStreamTracks. The specification also describes an API to generate
MediaStreams and make them available for rendering in a media element in
HTML5.

The production of a MediaStreamTrack depends on the underlying hardware and
software, which may introduce some latency between the time when the data is
detected to the time when it is made available to the Web application. The
specification requires user agents to expose the target latency for each
track.

The playback of a MediaStream is subject to the same considerations as those
raised above when discussing media support in HTML5. The media clock is
implementation-dependent in particular. Moreover, a MediaStream is a
\emph{live} element and is not seekable. The \emph{currentTime} and
\emph{playbackRate} properties of the media element that renders a MediaStream
are \emph{read-only} (i.e. media controls do not apply), and thus cannot be
adjusted for synchronization\footnote{In the future, it may be possible to re-create 
a seekable stream out of a MediaStream, thanks to the
\emph{MediaRecorder} interface defined in the MediaStream Recording
specification~\cite{mediastreamrecording}. This specification is not yet stable when this book is
written.}.


\subsection{WebRTC}
\label{sec:webrtc}

Work on \emph{Web Real-Time Communication (WebRTC)} and its first specification, 
the WebRTC 1.0: Real-time Communication Between Browsers
specification~\cite{webrtc}, started at the same time as the work on media
capture, in 2011. As the name suggests, the specification allows media and
data to be sent to and received from another browser. There is no fixed timing
defined, and the goal is to minimize latency. How this is achieved in practice
is up to the underlying protocols, which have been designed to reduce latency
and allow peer-to-peer communications between devices.

The WebRTC API builds on top of the Media Capture and Streams specification
and allows the exchange of MediaStreams. On top of the synchronization
restrictions noted above, a remote peer does not have any way to relate the
media timeline of the MediaStream it receives with the clock of the local
peer that sent it. The \mbox{WebRTC} API does not expose synchronization primitives.
This is up to Web applications, which may for instance exchange
synchronization parameters over a peer-to-peer data channel. Also, the
MediaStreamTracks that compose a MediaStream are essentially treated
independently and re-aligned for rendering on the remote peer, when possible.
In case of transmission errors or delays, loss of synchronization, e.g.
between audio and video tracks, is often preferred in WebRTC scenarios to
avoid accumulation of delays and glitches.

\subsection{Summary}

While the High Resolution Time clock is a step in the right direction, the
adoption is still incomplete. As of early 2017, given an arbitrary set of
timed data composed of audio/video content, animations, synthesized audio,
events, and more there are several issues Web developers need to face to
synchronize the presentation:


\begin{enumerate}

\item{
Clocks used by media components or media subsystems may be different and may
not follow the system clock. This is typically the case for media elements in
HTML5 and for the Web Audio API.
}

\item{
The clock used by a media component or a media subsystem may not be monotonic
or sufficiently precise.
}

\item{
Additionally, specifications may leave some leeway to implementers on the
accuracy of timed operations, leading to notable differences in behavior
across browsers.
}

\item{
Operations may introduce latencies that cannot easily be accounted for. This
includes running Web Animations, playing/resuming/capturing media, or
scheduling events on the event loop.
}

\item{
Standards may require browsers to pause a motion for buffering, as typically
happens for media playback in HTML5. This behavior does not play well with the
orchestration of video with other types of timed data that do not pause for
buffering.
}

\item{ The ability to relate clocks is often lost during the transmission of
timestamps from one place to another, either because different time origins
are used, as happens between an application and its workers, or because the
latency of the transmission is not accounted for, e.g. between WebRTC peers.
At best, applications developers need to use an out-of-band mechanism to
convert timestamps and account for the transport latency. }

\item{
When they exist, controls exposed to harness media components may not be
sufficiently fine-grained. For example, the \emph{playbackRate} property of media
elements in HTML5 was not designed for precise adjustments, and setting the
start time of a Web animation to a specific time value may result in a
significant jump between the first and second frames of the animation.
}

\end{enumerate}

Small improvements to Web technologies should resolve some of these issues,
and discussions are underway in relevant standardization groups at W3C when
this book is written. For example, timestamps in DOM Events may
switch to using the same \emph{Performance.now()} clock. This is all good news for
media synchronization, although it may still take time before the situation
improves.

We believe that a shift of paradigm is also needed. The Web is all about
modularity, composition and interoperability. Temporal aspects have remained
an internal issue specific to each technology until now. In the rest of this
chapter, a programming model is presented to work around the restrictions
mentioned above, allowing media to be precisely orchestrated on the Web, even
across devices.






% motion
\section{Motion}
\label{sec:motion}
%Motion is a simple concept representing playback state (media clock), as well
as functions for accessing and manipulating this state (media controls). As
such, similar constructs are found in most multimedia frameworks.

\begin{figure}[h]
%\sidecaption
\centering
\includegraphics[scale=.4]{fig/motion-axis.png}
\caption{Motion: point moving along an axis. The current position
is marked with a red circle (dashed), and forward velocity of 3 units per second is
indicated by the red arrow (dashed).}
\label{fig:motion}
\end{figure}

As illustrated in Fig.~\ref{fig:motion}, motion represents movement (in real-
time) of a point, along an axis (timeline). At any moment the point has well
defined position, velocity and acceleration\footnote{Some animation frameworks
support acceleration. Acceleration broadens the utility of motions, yet will
likely be ignored in common use cases in classical media (see
Sect.~\ref{sec:toomuch}).}. Velocity and acceleration describe continuous
movements. Velocity is defined as position-change per second, whereas
acceleration is defined as position- change per second squared. Discrete jumps
on the timeline are also supported, simply by modifying the position of the
motion. A discrete jump from position A to C implies that the transition took
no time, and that no position B (between A and C) was visited. Not moving
(i.e. zero velocity and acceleration) is a special case of movement.

\runinhead{Internal State.}
\label{sec:internalstate}
Motion is defined by an internal clock and a vector (position, velocity,
acceleration, timestamp). The vector describes the initial state of the
current movement, timestamped relative to the internal clock. This way, future
states of the motion may be calculated precisely from the initial vector and
elapsed time. Furthermore, application programmers may control the motion
simply by supplying a new initial vector. The motion concept was first
published under the name Media State Vector (MSV)~\cite{msv}.


\subsection{Timing object API}
\label{sec:motionapi}

Timing objects provides access to motions. Timing objects may be constructed
with a URL to an online motion. If the URL is omitted, it will represent a
local motion instead.

\begin{lstlisting}[caption=Constructing a timing object.]
var URL = "...";
var timingObject = new TimingObject(URL);
\end{lstlisting}

The \emph{Timing object API} defines two operations, \emph{query} and
\emph{update}, and emits a \emph{change} event as well as a periodic
\emph{timeupdate} event.


\runinhead{query():} 

The query operation returns a vector representing the current state of the
motion. This vector includes position, velocity and acceleration, as well as a
timestamp. For instance, if a query returns position 4.0 and velocity 1.0 and no acceleration, a
new query one second later will return position 5.0.

\begin{lstlisting}[caption=Querying the timing object to get a snapshot vector.]
var v = timingObject.query();
console.log("pos:" + v.position);
console.log("vel:" + v.velocity);
console.log("acc:" + v.acceleration);
\end{lstlisting}


\runinhead{update(vector):} 

The update operation accepts a vector parameter specifying new values for
position, velocity and acceleration. This initiates a new movement for the
motion. Omitting say position implies that the current position will be used.
So, an update with velocity 0 pauses the motion at the current position.

\begin{lstlisting}[caption=Updating the motion.]
// play, resume
timingObject.update({ velocity: 1.0 }); 
// pause
timingObject.update({ velocity: 0.0 });
// jump and play from 10
timingObject.update({ position: 10.0, velocity: 1.0});
// jump to position 10, keeping the current velocity
timingObject.update({ position: 10.0 });
\end{lstlisting}


\runinhead{change event:}

Whenever the motion is updated, all event listeners (i.e. media components)
will immediately be invoked. Note that the change event is not emitted
periodically like the \emph{timeupdate} event of HTML5 media elements. The change event signifies the start of a
new movement, not the continuation of a movement.

\runinhead{timeupdate event:}

For compatibility with existing HTML5 media elements and an easy way to update graphical elements, a \emph{timeupdate} evens is emitted periodically.

\begin{lstlisting}[caption=Monitoring changes to the motion through the change event.]
timingObject.on("change", function (e) {
  var v = motion.query();
  if (v.velocity === 0.0 && v.acceleration === 0.0) {
    console.log("I'm not moving!");
  } else {
    console.log("I'm moving!");
  }
});
\end{lstlisting}


\subsection{Programming with motions}

\runinhead{Online Motion:}

The Motion API is particularly designed to mediate access to online motions.
As discussed in Sect.~\ref{sec:model}, update operations are forwarded to the
online motion, and will not take effect until notification is received from
the online motion, and change events are emitted. In contrast, query is a local
(and cheap) operation. This ensures that media components may sample the
motion frequently if needed. In short, through the Motion API, online motions
are made available to Web developers as local objects. Only the latency of the
update operation is evidence of a distributed nature.

\runinhead{Using motions:}

Motions are resources used by Web applications, and the developer may define
as many as required. What purposes they serve in the application is up to the
programmer. If the motion should represent media offset in milliseconds, just
set the velocity to 1000 (advances the position of the motion by 1000
milliseconds per second). Or, for certain musical applications it may be
practical to let the motion represent beats per second.

\runinhead{Timing converters:}

A common challenge in media synchronization is that different sources of media
content may reference different timelines. For instance, one media stream may
have a logical timeline starting with 0, whereas another is timestamped with
epoch values. If the relation between these timelines is known (i.e.
\emph{relative skew}), it may be practical to create a skewed timing object
for one of the media components, connected to the motion. This is supported by
\emph{timing converters}. Multiple timing converters may be connected to a
motion, each implementing different transformations such as scaling and
looping. Timing converters may also be chained. Timign converters implement
the \emph{timing object API}, so media components can not distinguish between
a timing object and a timig converter. A number of timing converters are
implemented in the Timingsrc programming model~\cite{timingsrc}.

\runinhead{Flexibility:}
\label{sec:toomuch}

The mathematical nature of the motion concept makes it flexible, yet for a
particular media component some of this flexibility may be unnecessary, or
even unwanted. For instance, the HTML5 media player will typically not be able
to operate well with negative velocities, very high velocities, or with
acceleration. Fortunately, it does not have to. Instead, the media player may
define alternative modes of operation as long as the motion is in an
unsupported state. It could show a still image every second for high velocity,
or simply stop operation altogether (e.g. black screen with error message).
Later, when motion re-enters a supported state, normal operation may be
resumed for the media player.

%\subsection {Summary}
%Motion is a simple concept encapsulating media clock and media controls. It is
%designed particularly to be monitored by multiple media components, and as an
%interface towards online motions.


% motion sync
\section{Motion Synchronization}
\label{sec:motionsync}
%Temporal interoperability implies that multiple timed media components may
easily be combined into a single, consistent media experience. So far, we have
discussed this in the context of media components hosted within a single Web
page, or more accurately within a single browser context. However, the concept
of temporal interoperability also extends naturally to nested browsing
contexts (e.g. iframes), as well as multi-device media experiences. In short,
multi-device media require media components to provide consistent experiences
across processes and devices across Internet. To implement this, a transition
is needed from playback within a single browsing context to synchronized
playback in the distributed scenario.

\subsection {Shared Motion}

With the external motion approach, the challenge of synchronized multi-device
media is reduced to solving a single problem: distributed synchronization of
motion. Any solution to this problem is conceptualized as \emph{shared motion}. In
other words, synchronized, multi-device media is made from distributed media
components and shared motions.

The concept of shared motion is important, as it hides implementation detail
and emphasizes an attractive programming model for multi-device media
applications. In particular, by making synchronized motions available under
the same API as single-page motions (see Sect.~\ref{sec:model}), media components may be
applied in single-page as well as multi-device media experiences, without
modification. As such, the motion API provides much needed separation of
concern in multi-device media. By solving distributed motion synchronization
as a separate problem, application developers may focus on building great
media components using the motion API.

\runinhead{Technical Challenges:}


Precise, distributed motion synchronization requires a few technical
challenges to be addressed. In particular, two goals must be closely
approximated across all participating agents:

\begin{enumerate}
\item{a shared synchronized clock}
\item{a shared vector defining the current movement, timestamped in reference to the
synchronized clock}
\end{enumerate}

In addition, low latency is important for user experiences. Web agents should
be able to join synchronization quickly on page load or after page reload. To
achieve this, joining agents must quickly obtain the current vector and the
synchronized clock. For some applications the user experience might also
benefit from motion updates being disseminated quickly to all agents. Finally,
agents should be able to join and leave synchronization at any time, or fail,
without affecting the synchronization of other agents.


\subsection {A Server-based Approach}

While several approaches may be possible for motion synchronization, a server-
based approach seems like a natural fit for the Web domain. Motions would then
be represented as online resources (identifiable by URLs).

\begin{figure}[h]
%\sidecaption
\centering
\includegraphics[scale=.4]{fig/motion-sync.png}
\caption{The above figure illustrates 9 media components distributed across 3 different Web agents. Each Web agent uses a single motion to direct its media components. Given that the motions of the 3 different Web agents are precisely synchronized, it follows that all 9 media components will operate in precise synchrony.}
\label{fig:motion-sync}
\end{figure}

In Fig.~\ref{fig:motion-sync}, an online motion is hosted by an online motion server, and three
connected Web agents each maintains a local motion object precisely synchronized to the
online motion. This effectively means that the motion server manages motion
vectors. If one agent requests a motion vector to be updated, the server will
process this request and notify connected agents. Synchronized system clocks
(e.g. NTP) is generally not a valid assumption in the Web domain. As a
consequence, clock synchronization between agents and server must be
addressed. A clock server could be used by agents and motion server alike, or
this function could be integrated into the motion server. Cross-Internet clock
synchronization in JavaScript may be precise down to a few milliseconds
under normal network conditions, with probabilistic error bounds~\cite{msv, syncreport1, syncreport2}.

This server-based approach is attractive for a number of reasons. Centralized
services make distributed agreement and consistency easier to achieve. Also,
the client-server model naturally decouples synchronizing agents from each
other. This is particularly important in the Web domain, as Web clients may
disconnect or fail at any time. A server-based solution for shared motion also
implies that Web agents may quickly synchronize directly with the server, and
that motions will continue to exist across synchronization sessions, even if
no Web agents are connected. Finally, as a foundational component in multi-
device media, shared motions must be available anywhere and anytime the Web is
available. As such, implementing shared motion as an online service is likely
key to achieving global, cross-platform scope, high availability, reliability
and scalability.






% component sync
\section{Component Synchronization}
\label{sec:compsync}
%In the motion model, media components are parts of a larger media experience,
explicitly designed for external control. Component synchronization thus
involves a media component and an external motion object. The media component
synchronizes itself precisely relative to the motion. Relative synchronization
between media components connected to the same motion is achieved by
implication.

This section gives an introduction to component synchronization on the Web.
Synchronization of HTML5 media elements is covered. This serves as a case-
study for integrating established media frameworks into the motion model. In
addition, feasibility of the approach is demonstrated with respect to
synchronization quality. The section also covers development of custom media
components, focussing on synchronization (sequencing) of timed data (e.g.
subtitles, tracks, scripts, logs or time series). First though, component
synchronization is introduced by defining a basic pattern and some conventions.

\subsection{Basic Pattern}
\label{sec:pattern}

Consider a simple media component in a Web page. The media component has
access to two resources, media \textbf{data} and \textbf{motion}. Media data
is organized according to a timeline, and motion represents movement along
this timeline. Through its \textbf{UI} (user interface), the media component
may present \textbf{output} (e.g. pixels on screen, audio, vibration) and
receive \textbf{input} (e.g. key-presses, touch or mouse-events, camera, mic).

In these terms, the goal of component synchronization is simple: Given media
data and motion, generate the correct output at all times. To reach this goal,
implementations of component synchronization often follow a similar pattern.

\runinhead{Core processing loop:} 

Media components typically operate by repeating a core processing step:
\begin{enumerate}
\item{query the motion for current position (media offset)}
\item{identify the piece of media data that corresponds to current position}
\item{refresh the UI accordingly}
\end{enumerate}

For continuous media this core processing step is typically repeated at a
fixed frequency. For media components that visualize discrete media, the core
processing step is either run periodically (animation), or triggered by
timeout. In addition, three types of events interrupt the basic operation of
the media media component.

\runinhead{Motion change event:} 

Media components support media control by reacting to motion change events.
When position and/or velocity changes, internal state such as buffers and
schedules may have to be re-initialized, before the core processing step can
be run again. If paused (zero velocity), the media component runs its core
processing step once, and then terminate the core processing loop. If velocity
changes from zero to non-zero, the core processing loop must be started again.

\runinhead{Data change event:}

Some media components work with dynamic data sources, i.e. media data that may
change during presentation. This could be a live stream of timestamped
comments, or a time series from a live sensor. Media data may be added,
removed or modified. Modifications may imply re-positioning of data on the
timeline. In any case, media components must react by adjusting their internal
state accordingly, and run the core processing step.

\runinhead{Input (UI or API):} 

Interactive media components receive user input through UI events or API
methods. User input corresponding to media control requests must be forwarded
to the motion (causing motion change events by implication). This ensures that
control applies to all media components connected to the same motion. Other
types of user input trigger operations on the data model (causing data change
events by implication).

\subsection{Conventions}

In addition to the basic pattern, the motion model also defines a few
conventions for component synchronization.

\runinhead{Motion is the master:}

A distinctive feature of component synchronization is that the external motion
is not sensitive to the internal state of any media component. For instance,
motion might describe playback while a particular media component still lacks
data. If so, the media components should not halt the presentation until data
is ready. Instead, it must implement a reasonable behavior, given the
circumstances. For example, media components may adapt by buffering data
further ahead, changing to a different data source (e.g. lower bitrate) or
even changing to a different presentation mode (e.g audio only). This way,
playback may continue undisturbed and media components may join in as soon as
they are able to. This is particularly important in multi-device scenarios,
where a single device with limited bandwidth might otherwise hold back the
entire presentation. If the availability of a particular media component is
indeed essential to the media experience, this should instead be solved in
application-specific code, by pausing and resuming the external motion.

\runinhead{Reference point:}
\label{sec:referencepoint}
Media components synchronize their internal operations relative to an external
motion. However, if internal operations are subject to non-negligible delays
before they take effect, the resulting experience may still be incorrectly
synchronized. Though this problem is not specific to the motion model, it
still threatens the main ambition; that synchronization is simply a
consequence of connecting multiple media components to the same motion. To
avoid this, media components must compensate for delays internally by
scheduling actions earlier. This assumes that media components know or are
able to estimate such delays.

More formally, the concept of reference point in media playback refers to the
point in time when timed media data take physical effect. In the motion model,
the external motion defines the reference point.


\subsection {Integrating media frameworks}
\label{sec:html5sync}

Integration of existing media frameworks in the Web platform (e.g. HTML5 media
elements~\cite{html5media}, WebAudio~\cite{webaudio},
WebAnimation~\cite{webanimation}) into the motion model would be very
attractive. Currently though, they do not support internal synchronization, so
external synchronization is the only option. This means that synchronization
must be implemented in JavaScript, using the relevant control primitives that
each framework defines.

As a case-study in integration, this section discusses synchronization of
HTML5 media elements (i.e. audio and video). Media elements have not been
designed for synchronization, so this is not an optimal basis for evaluating
motion-based synchronization. Still, by identifying the capabilities and
limitations of current media elements, at least a baseline indication may be
provided. Likely, native support for synchronization will yield a significant
improvement.

The goal of motion-based synchronization of media elements is to keep the
\emph{currentTime} property (i.e. media offset) equal to the position of the
motion at all times, at least to a good approximation\footnote{This definition
assumes that the currentTime property matches the reference point for AV
playback. In our experience, this assumptions turns out to be reasonably
sound, though this is not mentioned as part of the specification for HTML5
media elements. }. The available control primitives include \emph{play},
\emph{pause}, \emph{seekTo} and \emph{variablePlaybackrate}. SeekTo specifies
a new media offset, and variablePlaybackrate allows media playback to speed up
or slow down.

The main challenge for precise synchronization is that HTML5 media elements
operate in their own time frame. They may be requested to play at a specific
time, yet the precise scheduling of media playback is subject to various
internal delays, such as time consumption in initialisation procedures,
buffering, decoding and AV-subsystems. There is also player drift, meaning
that the effective playback rate is not exactly equal to the rate of the
system clock. Finally, all these details vary across different architectures,
browsers and media formats.

So, the general approach is simply to evaluate currentTime property
continuously, and try to rectify whenever the synchronization error grows
beyond a certain threshold. For larger errors seekTo is used. This is
typically the case on load, or after motion change events. Smaller errors are
rectified gradually by manipulating variablePlaybackrate. SeekTo is quite
disruptive to the user experience, so support for variablePlaybackrate is
currently required for high quality synchronization.

\runinhead{Implementation:}

MediaSync is a JavaScript library (part of Timingsr~\cite{timingsrc}) allowing
HTML5 media elements to be synchronized by motion. The library works by
monitoring the motion, and adjusting the media element when needed. The
MediaSync library targets usage across the most common Web browsers, so it is
not optimized for any particular scenario.

The first practical challenge involves the resolution and precision of the
media offset. The currentTime property is updated at a fixed frequency,
typically about 4Hz, which is the recommended frequency for the timeupdate
event. The MediaSync library only samples the currentTime property immediately
after this event. More importantly, when compared to the system clock, the
currentTime property typically fluctuates considerably between samples. As a
consequence, the MediaSync library collects a backlog of samples, from which
the true value of the currentTime value can be estimated, with improved
resolution. Building up this backlog requires some samples, so it may take
more than a second for estimates to stabilize.

Another issue relates to unpredictable time-consumption in media control
operations. For instance, seekTo(X) will change currentTime to X, but it will
require a non-negligible amount of time to do this, as both buffering and
internal processing may be required. The problem is that seekTo does not
compensate for playback during its execution, so it is always wrong, at least
in the context of synchronization. In other words, it aims for a fixed target,
when it should aim for a moving target. The MediaSync library compensates for
this by overshooting the target. Furthermore, in order to overshoot by the
correct amount, the algorithm collects statistics from every seekTo operation.
Surprisingly perhaps, this strategy works.

Finally, there are some issues that cannot be compensated in JavaScript. Not
every browser supports variablePlaybackrate, and some claim they do even if it
is broken. Furthermore, sometimes playback is required to start at a media
offset other that 0. If so, synchronization starts with seekTo, followed by
play. In this case, the issue with time-consumption in seekTo implies that
buffering starts with data that will not be needed. This works against the
important goal of quickly being able to join a synchronized media
presentation.


\runinhead{Evaluation:}

Two technical reports~\cite{syncreport1,syncreport2} document the abilities and
limitations of HTML5 media elements with respect to media synchronization, as
well the quality of synchronization achieved by the MediaSync library. In
short, for desktops, laptops and high-end smartphones, echoless audio playback
is expected both for video+audio and audio only. Smartphones, and embedded
devices such a ChromeCast, can be expected to provide frame accurate
synchronization. Target precision is generally achieved within 3 seconds. It
is also worth noting that measured results are consistent with human
observation. Echoless synchronization with the MediaSync library produces
various audio effects, like failing to hear one audio source, until volume
levels are changed and only the other audio source can be heard. Since these
effects are also achieved across browser types and architectures, this is a
strong indication that currentTime approximates the reference point
quite well (see Sect.~\ref{sec:referencepoint}).

\begin{figure}[h]
%\sidecaption
\centering
\includegraphics[scale=.23]{fig/android-video.png}
\caption{The figure illustrates an experiment with video (mp4) synchronization on
Android using Chrome browser. The plot shows currentTime compared to the ideal
playback position defined by motion. The green band (echoless) is +-10 millisecond and
the yellow (frame accurate is) +-25 millisecond. The media element enters frame accurate
playback after two seekTo operations, and converges on echoless using
variablePlaybackrate.}
\label{fig:videosync}
\end{figure}

Though echoless synchronization is generally achievable, there are always a
few combinations of architecture, browser and media format that pose problems.
Errors may also occur and vanish across software updates. To be able to
support echoless synchronization reliably across the board, standards must
include requirements for synchronization, and testing-suites must be developed
to ensure that those requirements are met. Ideally though, media
synchronization should be implemented natively in media elements. Given the
challenges faced by the MediaSync library, it is not unlikely that native
implementations would provide a significant improvement.

\subsection{Custom media components}

Custom media components are developed for applications specific media sources
and/or applications specific visualizations. Building a custom media component
is basically a standard exercise in Web development. For instance, development
may involve a data source (e.g. JSON data), UI ($<div>$ element in the DOM),
and some code (JavaScript) implementing the functions of the basic  pattern
(see Sect.~\ref{sec:pattern}). Still, the basic pattern includes challenges
regarding timing and synchronization that may be challenging, especially for
developers with limited time on their hands. Fortunately, much of this
complexity may be encapsulated and solved by generic programming tools. In
this section we focus on custom media components working with timed data (e.g.
subtitles, tracks, scripts, logs, time series), and demonstrate  how a
sequencer turns this into a trivial programing challenge.

\runinhead{Sequencer:}

Timed data typically involves items tied to points or intervals on a timeline.
Synchronized presentation (or visualization) of timed data then requires items
to be activated and deactivated at the correct time, in reference to the
motion.

\begin{figure}[h]
%\sidecaption
\centering
\includegraphics[scale=.4]{fig/sequencer.jpg}
\caption{Five data sources of timed data, with items tied to intervals on the timeline. Motion along the same timeline defines which items are active (vertical dotted line), and precisely when items will be activated or deactivated.}
\label{fig:sequencer}
\end{figure}


The Sequencer~\cite{sequencer} is a generic tool for precise activation and
deactivation of timed data, in reference to a motion. Web developers parse
media data and registers cues tied to points or intervals on the timeline. The
Sequencer is then responsible for emitting enter and exit events for those
cues, at the correct points in time. By associating callback function to those
events, Web developers may update visualizations at the correct time.

The sequencer fully supports the basic pattern of component synchronization.
It is directed by motion and supports all the flexibility of motion-based
media control, including skipping, reverse playback and acceleration. It also
supports dynamic changes to data. Cues may safely be added, removed or changed
at any time during media presentation. The Sequencer guarantees that emitted
enter and exit events are always consistent with the current state of the
motion, and the current state of media data (i.e. cues.). By operating on cues
instead of data, the Sequencer does not require a specific data format. Also,
the Sequencer is not tied to any predefined visuals, leaving Web programmers
to implement synchronized UI’s with the full power of the Web platform at
hand.

In short, the Sequencer constitutes an important building block for
development of custom media components. It encapsulates complexity related to
motion-based synchronization and dynamic data, and leaves Web developers with
two simple task; registering cues and implementing event listeners. By using
generic programming tools such as the Sequencer,  Web developers with little
expertise in media synchronization may develop precisely synchronized
visualization based on application specific data sources.

\runinhead{Sequencer API:}
The following code snippet outlines the Sequencer API. The
complete API documentation, example code and demos are available at the
Timingsrc~\cite{timingsrc} Website.

\begin{lstlisting}[caption=Sequencer usage example]
// sequencer constructor
var seq = new Sequencer(motion);
// add or modify cue
seq.addCue("key", new Interval(12.2, 14.4), "data");
// remove cue
seq.removeCue("key");
// register event handler
var enter = function(e){console.log("enter", e.key, e.data)};
var exit = function(e){console.log("exit", e.key, e.data)};
seq.on('enter', enter);
seq.off('exit', exit)
\end{lstlisting}

\runinhead{Implementation:}

The Sequencer is implemented as a JavaScript library, and open-sourced as part
of the Timingsrc programming model~\cite{timingsrc}. In the interest of
precisely synchronized activation and deactivation and low cpu consumption,
the Sequencer implementation is not based on frequent polling. Instead, the
deterministic nature of motion allow events to be calculated and scheduled
using setTimeout, the timeout mechanism available in Web browsers. Though this
mechanism is not optimized for precision, Web browsers may be precise down to
a single millisecond. Over the last years, the Sequencer has been used
extensively by our research group. Further details are presented in this
paper~\cite{sequencer}.


%\subsection{Summary}
%The motion model makes media synchronization very easy for Web developers,
%especially if popular media frameworks have already been integrated and
%made available as reusable, common-purpose media components. If not,
%developers may have to implement wrapper code for media frameworks
%themselves, or implement custom media components. Sequencing
%tools makes development of custom media components much easier.



% evaluation
\section{Evaluation}
\label{sec:eval}
%The evaluation is concerned with feasibility of the motion model, as well as
it two primary objectives; Web availability and simplicity for developers.

Sect.~\ref{sec:motionsync} discussed motion synchronization, and reported
synchronizations errors in the 0-5 millisecond range. Typically we observe 0-1
millisecond errors for desktop browsers, compared to a system clock
synchronized by NTS. These results are achieved using the \emph{InMotion}
service provided by Motion Corporation~\cite{mcorp}. This service has been running
continuously for years, supporting a wide range of technical demonstrations,
at any time, at any place, and across a wide range of devices. As such, the
value of a production grade service is also confirmed.

Sect.~\ref{sec:compsync} discussed synchronization of HTML5 media
elements, and reported how the MediaSync library produces synchronization
errors of about 7 milliseconds for both audio and video, as documented in
technical reports~\cite{syncreport1,syncreport2}. These results have been consistently
confirmed by day to day usage over several years. The user experience of
multi-device video synchronization is very good, to the point that errors are
hardly visible. The exception might be content with particular geometries and
rapid movements, as demonstrated by this video~\cite{carneval}. Synchronization
has also been maintained for hours and days at end, without any visible
errors. In addition, loading speeds are acceptable. Even though the MediaSync
library requires about 3 seconds to reach echoless, the experience is
perceived as acceptable much before this. A variety of video demonstrations
have been published at the Multi-device Timing Community Group
Website~\cite{mtcg}.

Interestingly, the results for motion synchronization and HTML5 media
synchronization are well aligned with current limitations of the Web platform.
For instance, the precision of timed operation in JavaScript is about 1
millisecond, and a 60Hz screen refresh rate corresponds to 16 milliseconds.
Furthermore, these results also match limitations in human sensitivity to
synchronization errors. Identical audio signals skewed by less than 7
milliseconds will likely be interpreted as natural echo by the brain, and
collapsed into one signal (with directional information)~\cite{syncreport2}.

Finally, programming synchronized media experiences in the motion model is
both easy and rewarding. In our experience, motions and sequencers are
effective thinking tools as well as programming tools. A globally synchronized
video experience essentially requires three code statements.

With this, we argue that the feasibility of the motion model is confirmed. Web
availability for media synchronization is demonstrated by the \emph{InMotion} hosting
service for online motions. It is also clear that synchronization errors in
online synchronization are currently dominated by errors in synchronization in
HTML5 media elements. Future standardization efforts and optimizations would
likely yield significant improvements.


% standardization
\section{Standardization}
\label{sec:standard}
%The Web is widely regarded as a universal multi-media platform. Yet, it still
lacks a common model for timing and media control. The motion model promises
to fill this gap, and indicates a significant potential for the Web as a
platform for globally synchronized capture and playback of timed multimedia.
To bring these possibilities to the attention of the Web community, the motion
model has been proposed for Web standardization. The Multi-device Timing
Community Group (MTCG)~\cite{mtcg} has been created to attract support for
this initiative. The MTCG has published the draft specification for the 
TimingObject~\cite{timingobject}. It has also published Timingsrc~\cite{timingsrc}, an open
source JavaScript implementation of the TimingObject specification, including
timing objects, timing converters, sequencers and the mediaSync library.

Though the ideas promoted by the MTCG have been received with enthusiasm by
members within the W3C and within the wider Web community, at present the MTCG
proposal has not been evaluated by the W3C.


% conclusions
\section{Conclusions}
\label{sec:concl}
%In this chapter we presented the motion model, a general approach to media
synchronization on the Web. In the motion model, media clock and media
controls are merged into a single concept, motion, and made available as an
online resource. This ensures global synchronization and Web availability.
Media synchronization is simply a consequence of connecting multiple media
components to the same online motion. Furthermore, online motions are
represented in Web browsers by local motion objects. This way, media
synchronization is reduced to a local challenge, thereby shielding Web
developers from the complexities of distributed motion synchronization.
Existing media frameworks may be integrated into the motion model by
JavaScript wrapper code, yet internal support would likely be better.
Sequencing tools simplify the construction of custom media components,
effectively reducing the challenge of media synchronization to the challenge
of regular Web development.

The feasibility of the motion model is confirmed by evaluating synchronization
of HTML5 media elements connected to shared online motion. By this approach,
global, echoless audio and video synchronization is achievable across
different Web browsers, architectures and devices. The simplicity and broad
utility of the approach has also been confirmed through countless technical
demonstrations over the last few years, touching a wide range of use-cases in
digital multimedia media.

Importantly though, while quality of synchronization is key to determine
feasibility, it may not be the most relevant metric for the value of the
model. Likely, the extreme flexibility of the model is even more valuable. The
motion model allows complex multimedia experiences to be built by
synchronizing large numbers of dedicated, reusable media components,
specialized for a specific objective, data type and/or delivery method, across
devices globally, or within a single Web page. There is also great flexibility
in how media components may be added or removed dynamically, or perhaps
switched to a different motion. Furthermore, as any IP-connected device may
connect to an online motion, media experiences may cross platform boundaries
without even requiring any specific integration, perhaps making use of
dedicated visualization software or platform-specific content sources, while
remaining synchronized with Web-based media components.

Finally, modern multimedia products are facing increasingly tough
requirements. There are new data sources, new transfer protocols, new content
types and new visualization technologies. In addition, user demands are ever
rising; products must be immersive, personalized, interactive, exploit
secondary devices, and more. If media systems are not designed for
flexibility, extending them may be costly and time-consuming, eventually
leading to unmanageable complexity. In this context, the motion model shines.


% references
\section*{References}
\bibliographystyle{spmpsci}
\bibliography{chapter} 

\end{document}
