Proliferation of user devices makes synchronized, multi-device, multimedia
a reasonable expectation, and the Web platform should be a natural home
for such media experiences. Unfortunately though, it lacks a vital ingredient; support for multi-device 
media synchronization. This chapter presents the motion model, a
general approach to media synchronization on the Web. The motion model targets
simplicity and Web availability, emphasizing 1) that media synchronization
must be easy for Web developers, and 2) that media synchronization on the Web
must work wherever and whenever the Web works. In the motion model, media
clock and media controls are merged into a single concept, motion, and made
available as an online resource. Media synchronization is achieved by
connecting distributed media components to the same online motion. The
feasibility of this approach is confirmed by evaluating synchronization of
HTML5 media elements connected to online motion. Echoless synchronization of
audio and video is routinely produced across different Web browsers,
architectures and devices, making no assumptions except connectivity.
Importantly though, the value promise of the motion model is not limited to
synchronization alone. In particular, the extreme flexibility of the motion
model sets it up as a foundation for Web-based, multi-device, multimedia.
Though the motion model is designed explicitly for the Web, its utility is not
limited to the Web. Any connected device may connect to an online motion,
opening for synchronized media experiences across Web and native platforms.
The motion model has been suggested for Web standardization by the W3C multi-device 
timing community group.
