The proliferation of user devices as well as the eagerness of people to use
them, make seamless multi-device media experiences a natural vision in the
media industry. Such media experiences might for instance exploit big screens
for video content, whereas laptops, tablets and smartphones may be useful for
presenting associated content, such as infographics, backstory, foreign
language resources, control features or interactivity. People might also
expect much flexibility regarding the configuration of such experiences. Some
might meticulously define their own setups for a favourite event, whereas
others will prefer that media experiences automatically adapt to exploit
available devices. It must also be possible to shift media experiences
dynamically between devices, and adapt to new circumstances. For instance,
while driving, users on a strict network quota might want a live TV show to
adapt by dropping the video stream all toghether, while keeping the audio and
the secondary content. In addition, users always need the ability to control
their media sessions; e.g. pausing a live experience and resuming it at a more
convenient time, perhaps the following day on a different device. Importantly,
controls such as pause, rewind, fastforward or \emph{go to next scene} should
generally apply to all devices. In short, in this new multi-device world,
people wish to interact with media experiences through any device or multiple
devices, and media providers must create flexible media products capable of
exploiting these circumstances well. In particular, media products must support
radical transformations when circumstances change in significant ways. 


The Web platform appears to be a great match for this new multi-device world. It
is a powerful platform for multimedia, with global reach, powerful backend
services. Web browsers have a rich set of sensors for capture, interactive
capabilities, and a variety of frameworks for visualization and media
presentation. The modularity, interoperability and programmability of the Web
platform is also an excellent foundation for highly flexible media products.
Finally, the Web is already multi-device and global in nature. People may
interact with their online mailbox through any device, or using multiple
devices at the same time. Furthermore, they expect their online mailbox to be
the same, independent of which device is used to access it, or where they
might be at the time. Clearly, this basic promise of the Web is equally
attractive in the context of multi-device media experiences.
 
Unfortunately though, the Web lacks a crucial ingredient. Multi-device media
experiences (unlike mailboxes) require precise, distributed media
synchronization, and the Web platform provides no support for this. To fill
this gap, this chapter presents the motion model, a general approach to
media synchronization on the Web. The motion model is designed with the
following key objectives.

\begin{itemize}
\item{\emph{Global:} Media synchronization on the Web must be multi-device 
and global in nature, simply because the Web is multi-device and global in nature.}
\item{\emph{Web-availability:} If a Web browser is able to load a Web page, 
it should also be able to synchronize correctly.}
\item{\emph{Simplicity:} Media synchronization should be easy for Web developers. 
In particular, developers should be shielded from complexities of distributed synchronization.}
\end{itemize}

The key idea of the motion model is very simple: Media playback must be
represented as an online resource. This allows Web pages globally to
synchronize presentations relative to a shared playback state. Media
synchronization between Web pages is simply a consequence of connecting to the
same playback resource.

Though the primary purpose of the motion model is synchronization, the model
has implications that go far beyond this topic. In particular, the motion
model inverts the relationship between synchronization and media.
Traditionally, synchronization is external to media and may be applied to
media, if needed. In contrast, the motion model defines synchronization as an
integral part of media. As such, the motion model implies a new media model
particularly suited for highly flexible multi-device multimedia experiences.

The chapter is structured as follows: Sect.~\ref{sec:mediasync} defines media
synchronization as a term and briefly presents common challenges for synchronization on the Web. The motion
model is presented in Sect.~\ref{sec:model}, followed by
Sect.~\ref{sec:properties} which highlights key properties of the implied
media model. Sect.~\ref{sec:web} surveys relevant Web technologies, relating
them to the motion model.
Sect.~\ref{sec:motion},~\ref{sec:motionsync} and~\ref{sec:compsync} give a more
thorough introduction to the motion concept, to distributed synchronization of
motions, and to the challenge of synchronizing media components. Evaluation is
presented in Sect.~\ref{sec:eval}. Sect.~\ref{sec:standard} briefly references
the standardization initiative made by the Multi-device Timing Community
Group~\cite{mtcg}, before conclusions are given in Sect.~\ref{sec:concl}.
